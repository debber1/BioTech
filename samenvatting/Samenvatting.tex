% !TeX spellcheck = nl_NL
\documentclass[a4paper,kul]{kulakarticle} %options: kul or kulak (default)

\usepackage[utf8]{inputenc}
\usepackage[dutch]{babel}

\date{Academiejaar 2021 -- 2022}
\address{
	Industriële Ingenieurswetenschappen \\
	BioTechnologie \\
	Inge Holsbeeks \& Hans Rediers}
\title{Samenvatting}
\author{Robbe Decapmaker}
\usepackage{hyperref}
\usepackage{graphicx}
\usepackage{amsmath, amssymb, amsthm}
\usepackage{siunitx}
\usepackage{flafter} 
\usepackage{pdfpages}
\usepackage{caption}
\usepackage{subcaption}



\begin{document}

\maketitle

\section*{Inleiding}

De samenvatting van BioTechnologie. \href{https://github.com/debber1/BioTech}{De source code is te vinden op Github.}\\
%DEZE ZIN IS ENKEL RELEVANT TIJDENS DE ONTWIKKELING VAN DIT DOCUMENT
\textbf{Dit document is een `work in progress', dit wil zeggen dat er (ongeveer) een wekelijkse update zal zijn. De meest recente versie zal altijd op Github staan!}
\tableofcontents
\newpage
\section{Koolhydraten}
Koolhydraten zijn essentieel voor biologisch leven. Grosso modo kunnen we 3 verschillende types onderscheiden: monosachariden, disachariden en polysachariden. Voor dat we deze types degelijk kunnen bespreken moet er eerst enkele afspraken vast gelegd worden rond naamgeving een voorstelling. We moeten ook nog enkele belangrijke opmerken maken rond de chemische fenomenen die zich voor doen bij koolhydraten. 
\subsection{Naamgeving}
Koolhydraten bestaan voornamelijk uit C, O en H atomen. Afhankelijk van de onderling gevormde bindingen kunnen we een onderscheid maken tussen twee soorten koolhydraten; de aldosen en ketonen. Het verschil tussen beiden wordt duidelijk gemaakt in figuur \ref{fig:aldehyde-keton}. Als een koolhydraat in bezit is van een aldehyde groep, noemen we hem een aldose. Als hij in bezit is van een keton groep, noemen we hem een ketose.
\begin{figure}[htbp]
	\centering
	\includegraphics[width=0.7\linewidth]{Aldehyde-Keton}
	\caption[Aldehyden en ketonen]{Aldehyden en ketonen}
	\label{fig:aldehyde-keton}
\end{figure}\\
Naast de aanwezigheid van functionele groepen, maken we ook een onderscheid op basis van het aantal aanwezige koolstof atomen. De nummering en naamgeving van deze moleculen worden overgenomen uit de chemie zoals te zien is op figuur \ref{fig:examples-name}.
\begin{figure}[htbp]
	\centering
	\includegraphics[width=0.6\linewidth]{examples-name}
	\caption[Naamgeving]{Voorbeelden van naamgeving}
	\label{fig:examples-name}
\end{figure}\\
Er zijn ook enkele koolhydraten die een triviale naam krijgen, zoals sacharose of fructose.

\subsection{Voorstellingen}
Er bestaan twee manieren om een koolhydraat voor te stellen, de Fischer- en Haworthprojectie. Voor D-glucose zien we op figuur \ref{fig:fishervshaworth} beide voorstellingen.
\begin{figure}[h]
	\centering
	\includegraphics[width=0.6\linewidth]{FisherVSHaworth}
	\caption[Fischer- en Haworthprojectie]{Fischerprojectie (links) en Haworthprojectie (rechts)}
	\label{fig:fishervshaworth}
\end{figure}

\subsection{Stereochemie}
Als de structuur van een koolhydraat koolstof atomen bevat die gebonden zijn met vier verschillende groepen, zeggen we dat de structuur een chiraal centrum heeft. Dit fenomeen kan tot opmerkelijke resultaten leiden, zo is het mogelijk da bepaalde functionele groepen niet altijd op dezelfde manier georiënteerd zijn.  
\subsubsection{Enantiomeren}
We spreken van enantiomeren als een we te maken hebben met een molecule die volledig gespiegeld kan worden. Een voorbeeld is te zien op figuur \ref{fig:enantiomeren}. Deze spiegeling heeft enkele gevolgen, zowel op biologisch als op fysisch vlak. Zo kunnen verschillende enantiomeren anders reageren op gepolariseerd licht. Vanuit een biologisch standpunt vormt er een probleem als de enantiomeren niet op dezelfde manier samenwerken met enzymen (zie figuur \ref{fig:enantiomeerenzym}). Als beide enantiomeren (normaal en gespiegeld of L en D in een biologische context) aanwezig zijn in een mengsel, dan nomen we dit een racemisch mengsel.
Het is ook belangrijk om op te merken dat een gespiegelde tekening niet zomaar een enantiomeer is. Het is ook mogelijk dat er een mesoverbinding aan het werk is. Dit is een verbinding met twee of meer chirale koolstofatomen en een intern symmetrievlak zoals te zien is op figuur \ref{fig:mesoverbinding}. 
\begin{figure}
	\centering
	\includegraphics[width=0.7\linewidth]{mesoverbinding}
	\caption[Mesoverbinding]{Enantiomeer (links) en mesoverbinding (rechts)}
	\label{fig:mesoverbinding}
\end{figure}

\begin{figure}[htbp]
	\centering
	\includegraphics[width=0.6\linewidth]{enantiomeren}
	\caption[Enantiomeer]{Enantiomeer}
	\label{fig:enantiomeren}
\end{figure}
\begin{figure}[htbp]
	\centering
	\includegraphics[width=0.5\linewidth]{EnantiomeerEnzym}
	\caption[Enantiomeer en enzym]{Interactie tussen enantiomeren en enzymen}
	\label{fig:enantiomeerenzym}
\end{figure}


\subsubsection{Diastereomeren}
\begin{figure}[htbp]
	\centering
	\includegraphics[width=0.4\linewidth]{Diastereomeren}
	\caption[Diastereomeren]{Diastereomeren}
	\label{fig:diastereomeren}
\end{figure}
Diastereomeren zijn zoals enantiomeren, maar ze zijn geen perfect spiegelbeeld zoals te zien is op figuur \ref{fig:diastereomeren}.
\subsubsection{Glucose}
Het bekendste voorbeeld van deze fenomenen is glucose. In de natuur observeren we D-glucose en L-glucose. Hiervan zien we hoofdzakelijk D-glucose voorkomen omdat dit het type glucose is dat gemaakt wordt door fotosynthese. Ons lichaam maakt wel een onderscheid tussen beide varianten, ze smaken alle twee zoet maar enkel D-glucose heeft een calorische inhoud bij het verteren. Dit wil zeggen dat L-glucose niet wordt opgenomen door ons spijsverterings-stelsel, en dus niet kan  gebruikt worden om energie uit te halen. Het is dus een `zoetstof'.

\subsection{Reducerende koolhydraten}
We spreken van een reducerend koolhydraat als het molecuul optreedt als reducerend agens in een reactie als gevolg van de aanwezigheid van de aldehyde- of ketongroep (zie figuur \ref{fig:reductiefding}). Veel monosachariden bezitten deze eigenschap, daarnaast hebben ook disachariden, waarvan het anomere koolstof-atoom geen glycoside binding heeft, ook een reducerend vermogen. Polysachariden hebben meestal een te lage hoeveelheid reducerende uiteinden om een reductief karakter te hebben.
\begin{figure}[htbp]
	\centering
	\includegraphics[width=0.5\linewidth]{ReductiefDing}
	\caption[Reducerende koolhydraten]{Reducerende koolhydraten}
	\label{fig:reductiefding}
\end{figure}\\
We kunnen testen of een koolhydraat in het bezit is van een reducerend karakter met Benetict's reagens. Hierbij kijken we naar de kleurverandering van de oplossing na de reactie. Dit is tevens de manier waarop we kunnen bepalen of er suiker in urine zit, het is handig om suikerziekte op te sporen.

\subsection{Monosachariden}
Monosachariden zijn de meest eenvoudige koolhydraten. Ze vormen hierdoor dus ook de bouwstenen voor complexere structuren zoals disachariden en polysachariden. 
\subsubsection{Belangrijke monosachariden}
\textbf{Glucose}
\begin{figure}[h]
	\centering
	\includegraphics[width=0.7\linewidth]{GlucoseAlphaBeta}
	\caption[Glucose]{Glucose}
	\label{fig:glucosealphabeta}
\end{figure}\\
Hierbij kunnen we nog vermelden dat koolstof atoom 1 in figuur \ref{fig:glucosealphabeta} een nieuw chiraal centrum is, en dus een anomeer C-atoom is.\\
\textbf{Fructose}
\begin{figure}[h]
	\centering
	\includegraphics[width=0.4\linewidth]{FructoseAlphaBeta}
	\caption[Fructose]{Fructose}
	\label{fig:fructosealphabeta}
\end{figure}\\
Hierbij kunnen we nog vermelden dat koolstof atoom 2 in figuur \ref{fig:fructosealphabeta} een nieuw chiraal centrum is, en dus een anomeer C-atoom is.\\
\newpage
\textbf{Deoxyribose}
\begin{figure}[h]
	\centering
	\includegraphics[width=0.4\linewidth]{deoxyribosealphabeta}
	\caption[Deoxyribose]{Deoxyribose}
	\label{fig:deoxyribosealphabeta}
\end{figure}\\
Hierbij kunnen we nog vermelden dat koolstof atoom 1 in figuur \ref{fig:deoxyribosealphabeta} een nieuw chiraal centrum is, en dus een anomeer C-atoom is. Deoxyribose is tevens belangrijk voor RNA en DNA.\\
\textbf{Acetylglucosamine}\\
Door andere functionele groepen toe te voegen aan de koolhydraatstructuur kunnen we complexere moleculen maken. Acetylglucosamine (figuur \ref{fig:acetylglucosamine}) is bijvoorbeeld een bloed antigen.
\begin{figure}[h]
	\centering
	\includegraphics[width=0.4\linewidth]{Acetylglucosamine}
	\caption[Acetylglucosamine]{Acetylglucosamine}
	\label{fig:acetylglucosamine}
\end{figure}
\subsubsection{Afgeleiden}
Enkele voorbeelden van afgeleiden van monosachariden zijn polyolen. Ze zijn geen monosachariden, maar lijken er wel sterk op. De voorbeelden uit figuur \ref{fig:afgeleidenmonosachariden} zijn zoet zoals glucose, ze hebben wel geen calorie-inhoud. Ze zijn dus geschikt om zoetstoffen mee te maken.
\begin{figure}[h]
	\centering
	\includegraphics[width=0.4\linewidth]{AfgeleidenMonosachariden}
	\caption[Afgeleiden]{Afgeleiden van monosachariden}
	\label{fig:afgeleidenmonosachariden}
\end{figure}

\subsection{Disachariden}
Disachariden zijn opgebouwd uit 2 monosachariden. We kunnen ze benoemen volgens de monosachariden waaruit ze zijn opgebouwd, en de manier waarop deze gebonden zijn aan elkaar.
\subsubsection{Belangrijke disachariden}
\textbf{Sucrose}\\
Sucrose is wat wij kennen als 'gewoon' suiker. Het is opgebouwd uit $\alpha$-glucose en $\beta$-Fructose (zie figuur \ref{fig:sucrose}).
\begin{figure}[h]
	\centering
	\includegraphics[width=0.5\linewidth]{sucrose}
	\caption[Sucrose]{Sucrose}
	\label{fig:sucrose}
\end{figure}\\
\textbf{Lactose}\\
Lactose of melksuiker is opgebouwd uit $\beta$-D-galactose en $\beta$-D-glucose (zie figuur \ref{fig:lactose}).
\begin{figure}[h]
	\centering
	\includegraphics[width=0.6\linewidth]{Lactose}
	\caption[Lactose]{Lactose}
	\label{fig:lactose}
\end{figure}\\
\textbf{Maltose}\\
Maltose of moutsuiker is opgebouwd uit $\alpha$-D-glucose en $\beta$-D-glucose (zie figuur \ref{fig:maltose}).
\begin{figure}[h]
	\centering
	\includegraphics[width=0.6\linewidth]{Maltose}
	\caption[Maltose]{Maltose}
	\label{fig:maltose}
\end{figure}

\subsection{Polysachariden}
Polysachariden zijn net zoals disachariden opgebouwd uit monosachariden. Het verschil zit in de hoeveelheid bouwblokken er aanwezig zijn. Grosso modo zullen we spreken over polysachariden als er meer dan 2 monosachariden betrokken zijn bij de opbouw.

Afhankelijk van de onderlinge bindingen, kunnen er zich vertakkingen voor doen in een polysacharide. beide polysachariden in figuur \ref{fig:vertakkingen} zijn opgebouwd volgens de structuur in figuur \ref{fig:vertakking}. Door het aanzienlijke verschil in vertakkingen, maken we een onderscheid tussen deze twee moleculen.
\begin{figure}[h]
	\centering
	\includegraphics[width=0.7\linewidth]{Vertakking}
	\caption[Vertakking]{Verschillende bindingen zorgen voor vertakkingen}
	\label{fig:vertakking}
\end{figure}

\begin{figure}[h]
	\centering
	\begin{subfigure}{.5\textwidth}
		\centering
		\includegraphics[width=.4\linewidth]{amylopectine_in_zetmeel.png}
		\caption{Amylopectine in zetmeel}
		\label{fig:sub1}
	\end{subfigure}%
	\begin{subfigure}{.5\textwidth}
		\centering
		\includegraphics[width=.7\linewidth]{Glycogeen.png}
		\caption{Glycogeen}
		\label{fig:sub2}
	\end{subfigure}
	\caption{Het verschil tussen veel en weinig vertakkingen}
	\label{fig:vertakkingen}
\end{figure}
\newpage
\subsubsection{Belangrijke polysachariden}
\textbf{Zetmeel en glycogeen}\\
Zetmeel en glycogeen lijken sterk op elkaar, ze zijn beiden namelijk opgebouwd volgens figuur \ref{fig:vertakking}. Zetmeel vinden we voornamelijk terug in planten, het is een belangrijke nutriënt voor de mens. We gebruiken het namelijk vaak voor het maken van brood en pasta. Een bijkomend voordeel, is dat het gemakkelijk afbreekbaar is tijdens de vertering. Glycogeen komt hoofdzakelijk voor bij dieren. Het is in grote concentraties aanwezig in spierweefsel en de lever. Dit is ook de meer vertakte variant van amylopectine (zie figuur \ref{fig:vertakkingen}).\\
\textbf{Cellulose}\\
Cellulose is vaak aanwezig bij planten. Het is niet verteerbaar door de mens, dit wil zeggen dat er eigenlijk geen calorische inhoud is. Het kan vaak gebruikt worden als een structureel component om een cel op te bouwen omdat het een soort vezel vormt. Dit is ook de reden waarom we cellulose gebruiken om papier te maken. De structuur van cellulose is te zien in figuur \ref{fig:cellulose}.
\begin{figure}[h]
	\centering
	\includegraphics[width=0.7\linewidth]{Cellulose}
	\caption[Cellulose]{Cellulose}
	\label{fig:cellulose}
\end{figure}

\section{Lipiden}
\begin{figure}[htbp]
	\centering
	\includegraphics[width=0.7\linewidth]{Schema_lipiden}
	\caption[Lipiden]{Schema van lipiden}
	\label{fig:schemalipiden}
\end{figure}

%Overzicht Schema slide 28
\subsection{Biologische functies van lipiden}
Biologisch gezien zijn vetten extreem belangrijk. De mens gebruikt ze namelijk voor verschillende doeleinden:
\begin{itemize}
	\item Energiebron en -opslag
	\item Structurele componenten van het celmembraan
	\item Hormonen
	\item Vitaminen en vitamine-adsorptie
	\item Bescherming
	\item Isolatie
\end{itemize}
\subsection{Vetzuren}
\subsubsection{Structuur}
Vetzuren hebben lange ketens van monocarboxylzuren (-COOH) met een even aantal koolstofatomen (zie figuur \ref{fig:structuurvetzuur}).
\begin{figure}[h]
	\centering
	\includegraphics[width=0.7\linewidth]{StructuurVetZuur}
	\caption[Vetzuur]{Structuur van een vetzuur}
	\label{fig:structuurvetzuur}
\end{figure}
\subsubsection{(On)Verzadigde vetzuren}
Verzadigde vetzuren bestaan uitsluitend uit enkelvoudig gebonden koolstof atomen zoals in figuur \ref{fig:structuurvetzuur}. We spreken van een onverzadigd vetzuur als er een dubbele binding voorkomt tussen de koolstoffen zoals in figuur \ref{fig:onverzadigdevetzuren}.
\begin{figure}[h]
	\centering
	\includegraphics[width=0.7\linewidth]{OnVerzadigdeVetzuren}
	\caption[Onverzadigde VZ]{Onverzadigde vetzuren}
	\label{fig:onverzadigdevetzuren}
\end{figure}\\
We vinden verzadigde vetzuren vaak bij dieren en onverzadigde vetzuren bij planten. Ook het smeltpunt kent grote verschillen tussen beide soorten (zie figuur \ref{fig:smeltpuntvetzuren}). Grosso modo kunnen we zeggen dat het smeltpunt van verzadigde vetzuren afhangt van de lengte van de koolstofketen (London krachten). Bij onverzadigde vetzuren is het smeltpunt invers proportioneel aan het aantal onverzadigde bindingen. 
\begin{figure}[h]
	\centering
	\includegraphics[width=0.7\linewidth]{smeltpuntVetzuren}
	\caption[Smeltpunt]{Smeltpunt van verzadigde en onverzadigde vetzuren}
	\label{fig:smeltpuntvetzuren}
\end{figure}
\subsubsection{Cis- en Transvetzuren}
Het verschil tussen cis- en transvetzuren zit in de manier waarop de koolstof atomen onderling georiënteerd zijn (zie figuur \ref{fig:cistransvz}). Over het algemeen kunnen we zeggen dat Transvetzuren slecht zijn voor de mens, ze hebben een negatief effect op zaken zoals cholesterol en hart- en vaatziekten. We vinden ze vaak bij producten gemaakt van herkauwers, en na verhitting van vetzuren. Cisvetzuren daarentegen hebben een positieve invloed op cholesterol.
\begin{figure}[h]
	\centering
	\includegraphics[width=0.7\linewidth]{CisTransVZ}
	\caption[Cis en Trans]{Cis- en Transvetzuren}
	\label{fig:cistransvz}
\end{figure}
\newpage
\subsubsection{Omega vetzuren}
We spreken hoofdzakelijk over Omega 3 en omega 6 vetzuren. Het zijn onverzadigde vetzuren die hun naam krijgen op basis van het aantal verzadigde bindingen voor de eerste dubbele binding, zie figuur \ref{fig:omegaVZ}. 
\begin{figure}[h]
	\centering
	\begin{subfigure}{.5\textwidth}
		\centering
		\includegraphics[width=1\linewidth]{Omega3VZ}
		\caption{Omega 3 vetzuur}
		\label{fig:Omega3}
	\end{subfigure}%
	\begin{subfigure}{.5\textwidth}
		\centering
		\includegraphics[width=1\linewidth]{Omega6VZ}
		\caption{Omega 6 vetzuur}
		\label{fig:Omega6}
	\end{subfigure}
	\caption{Het verschil tussen veel en weinig vertakkingen}
	\label{fig:omegaVZ}
\end{figure}\\
We associëren omega 3 vetzuren vaak met positieve gezondheidseffecten, onderzoek wijst namelijk uit dat het hart- en vaatziekten tegenhoud en een ontstekingsremmend effect heeft. Omega 6 vetzuren komen met ongewenste gezondheidseffecten.
\subsubsection{Reacties met vetzuren}
De belangrijkste reactie voor deze cursus is de hydrogenering (zie figuur ). Het is een additie reactie waarbij onverzadigde vetzuren omgezet worden in verzadigde vetzuren. Deze reactie wordt sterk gebruikt in de voedingsindustrie.
\begin{figure}[h]
	\centering
	\includegraphics[width=0.7\linewidth]{hydrogenering}
	\caption[Hydrogenering]{Hydrogenering}
	\label{fig:hydrogenering}
\end{figure}
%Mooie samenvatting op slide 16
\subsection{Glyceriden}
\subsubsection{Structuur}
Glyceriden zijn lipide-esters, dit wil zeggen dat de alcoholgroep van glycerol een ester vormt met een vetzuur (zie figuur \ref{fig:structuurglyceriden}). Deze estervorming kan zich op één, twee of alle 3 de alcoholgroepen van de glycerol voordoen. We spreken van mono-, di-, triglyceriden.
\begin{figure}[h]
	\centering
	\includegraphics[width=0.6\linewidth]{structuur_glyceriden}
	\caption[Structuur glyceriden]{Structuur glyceriden}
	\label{fig:structuurglyceriden}
\end{figure}

\subsubsection{Triglyceriden}
\subsubsection{Reacties}
\subsubsection{Fosfoglyceriden}
\subsection{Niet-glyceride lipiden}
\subsubsection{Sfingolipide}
\subsubsection{Steroïden}




\end{document}
