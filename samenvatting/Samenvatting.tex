% !TeX spellcheck = nl_NL
\documentclass[a4paper,kul]{kulakarticle} %options: kul or kulak (default)

\usepackage[utf8]{inputenc}
\usepackage[dutch]{babel}

\date{Academiejaar 2021 -- 2022}
\address{
	Industriële Ingenieurswetenschappen \\
	BioTechnologie \\
	Inge Holsbeeks \& Hans Rediers}
\title{Samenvatting}
\author{Robbe Decapmaker}
\usepackage{hyperref}
\usepackage{graphicx}
\usepackage{amsmath, amssymb, amsthm}
\usepackage{siunitx}
\usepackage{flafter} 
\usepackage{pdfpages}



\begin{document}

\maketitle

\section*{Inleiding}

De samenvatting van BioTechnologie. \href{https://github.com/debber1/BioTech}{De source code is te vinden op Github.}\\
%DEZE ZIN IS ENKEL RELEVANT TIJDENS DE ONTWIKKELING VAN DIT DOCUMENT
\textbf{Dit document is een `work in progress', dit wil zeggen dat er (ongeveer) een wekelijkse update zal zijn. De meest recente versie zal altijd op Github staan!}
\tableofcontents
\newpage
\section{Koolhydraten}
Koolhydraten zijn essentieel voor biologisch leven. Grosso modo kunnen we 3 verschillende types onderscheiden: monosachariden, disachariden en polysachariden. Voor dat we deze types degelijk kunnen bespreken moet er eerst enkele afspraken vast gelegd worden rond naamgeving een voorstelling. We moeten ook nog enkele belangrijke opmerken maken rond de chemische fenomenen die zich voor doen bij koolhydraten. 
\subsection{Naamgeving}
Koolhydraten bestaan voornamelijk uit C, O en H atomen. Afhankelijk van de onderling gevormde bindingen kunnen we een onderscheid maken tussen twee soorten koolhydraten; de aldosen en ketonen. Het verschil tussen beiden wordt duidelijk gemaakt in figuur \ref{fig:aldehyde-keton}. Als een koolhydraat in bezit is van een aldehyde groep, noemen we hem een aldose. Als hij in bezit is van een keton groep, noemen we hem een ketose.
\begin{figure}[htbp]
	\centering
	\includegraphics[width=0.7\linewidth]{Aldehyde-Keton}
	\caption[Aldehyden en ketonen]{Aldehyden en ketonen}
	\label{fig:aldehyde-keton}
\end{figure}\\
Naast de aanwezigheid van functionele groepen, maken we ook een onderscheid op basis van het aantal aanwezige koolstof atomen. De nummering en naamgeving van deze moleculen worden overgenomen uit de chemie zoals te zien is op figuur \ref{fig:examples-name}.
\begin{figure}[htbp]
	\centering
	\includegraphics[width=0.6\linewidth]{examples-name}
	\caption[Naamgeving]{Voorbeelden van naamgeving}
	\label{fig:examples-name}
\end{figure}\\
Er zijn ook enkele koolhydraten die een triviale naam krijgen, zoals sacharose of fructose.

\subsection{Voorstellingen}
Er bestaan twee manieren om een koolhydraat voor te stellen, de Fischer- en Haworthprojectie. Voor D-glucose zien we op figuur \ref{fig:fishervshaworth} beide voorstellingen.
\begin{figure}[h]
	\centering
	\includegraphics[width=0.6\linewidth]{FisherVSHaworth}
	\caption[Fischer- en Haworthprojectie]{Fischerprojectie (links) en Haworthprojectie (rechts)}
	\label{fig:fishervshaworth}
\end{figure}

\subsection{Stereochemie}
Als de structuur van een koolhydraat koolstof atomen bevat die gebonden zijn met vier verschillende groepen, zeggen we dat de structuur een chiraal centrum heeft. Dit fenomeen kan tot opmerkelijke resultaten leiden, zo is het mogelijk da bepaalde functionele groepen niet altijd op dezelfde manier georiënteerd zijn.  
\subsubsection{Enantiomeren}
We spreken van enantiomeren als een we te maken hebben met een molecule die volledig gespiegeld kan worden. Een voorbeeld is te zien op figuur \ref{fig:enantiomeren}. Deze spiegeling heeft enkele gevolgen, zowel op biologisch als op fysisch vlak. Zo kunnen verschillende enantiomeren anders reageren op gepolariseerd licht. Vanuit een biologisch standpunt vormt er een probleem als de enantiomeren niet op dezelfde manier samenwerken met enzymen (zie figuur \ref{fig:enantiomeerenzym}). Als beide enantiomeren (normaal en gespiegeld of L en D in een biologische context) aanwezig zijn in een mengsel, dan nomen we dit een racemisch mengsel.
Het is ook belangrijk om op te merken dat een gespiegelde tekening niet zomaar een enantiomeer is. Het is ook mogelijk dat er een mesoverbinding aan het werk is. Dit is een verbinding met twee of meer chirale koolstofatomen en een intern symmetrievlak zoals te zien is op figuur \ref{fig:mesoverbinding}. 
\begin{figure}
	\centering
	\includegraphics[width=0.7\linewidth]{mesoverbinding}
	\caption[Mesoverbinding]{Enantiomeer (links) en mesoverbinding (rechts)}
	\label{fig:mesoverbinding}
\end{figure}

\begin{figure}[htbp]
	\centering
	\includegraphics[width=0.6\linewidth]{enantiomeren}
	\caption[Enantiomeer]{Enantiomeer}
	\label{fig:enantiomeren}
\end{figure}
\begin{figure}[htbp]
	\centering
	\includegraphics[width=0.5\linewidth]{EnantiomeerEnzym}
	\caption[Enantiomeer en enzym]{Interactie tussen enantiomeren en enzymen}
	\label{fig:enantiomeerenzym}
\end{figure}


\subsubsection{Diastereomeren}
\begin{figure}[htbp]
	\centering
	\includegraphics[width=0.4\linewidth]{Diastereomeren}
	\caption[Diastereomeren]{Diastereomeren}
	\label{fig:diastereomeren}
\end{figure}
Diastereomeren zijn zoals enantiomeren, maar ze zijn geen perfect spiegelbeeld zoals te zien is op figuur \ref{fig:diastereomeren}.

\subsection{Reducerende koolhydraten}
\subsection{Monosachariden}
\subsubsection{Belangrijke monosachariden}
\subsubsection{Afgeleiden}
\subsection{Disachariden}
\subsubsection{Belangrijke disachariden}
\subsection{Polysachariden}
\subsubsection{Belangrijke polysachariden}
\section{Lipiden}
%Overzicht Schema slide 28
\subsection{Biologische functies van lipiden}
\subsection{Vetzuren}
\subsubsection{Structuur}
\subsubsection{(On)Verzadigde vetzuren}
\subsubsection{Cis- en Transvetzuren}
\subsubsection{Omega vetzuren}
\subsubsection{Reacties met vetzuren}
%Mooie samenvatting op slide 16
\subsection{Glyceriden}
\subsubsection{Structuur}
\subsubsection{Triglyceriden}
\subsubsection{Reacties}
\subsubsection{Fosfoglyceriden}
\subsection{Niet-glyceride lipiden}
\subsubsection{Sfingolipide}
\subsubsection{Steroïden}




\end{document}
